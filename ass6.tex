\documentclass{article}
\usepackage[utf8]{inputenc}
\usepackage{karnaugh-map}

% ADD TITLE HERE
\title{K-map fore}
\author{SASHANK REDDY}
\date{4-12-2020}

\begin{document}

\maketitle

\section{kmap for e with with expression e=A\bar{D} + \bar{B}C\bar{D} + A\bar{C}\bar{B}$ \\ }

\begin{karnaugh-map}[4][4][1][][]
    \maxterms{0,2,6,8,10,11,12,13,14,15}
    \minterms{1,3,4,5,7,9}
   
    \implicant{12}{14}
      \implicant{2}{10}
    \implicant{15}{10}
    \implicantcorner{0}{2}{18}{10}
    % note: position for start of \draw is (0, Y) where Y is
    % the Y size(number of cells high) in this case Y=2
    \draw[color=black, ultra thin] (0, 4) --
    node [pos=0.7, above right, anchor=south west] {$XW$} % YOU CAN CHANGE NAME OF VAR HERE, THE $X$ IS USED FOR ITALICS
    node [pos=0.7, below left, anchor=north east] {$ZY$} % SAME FOR THIS
    ++(135:1);
        
    \end{karnaugh-map}
    
    
    
From the K-Map above, we can infer that the POS expression is:

\begin{equation}
    e=A\bar{D} + \bar{B}C\bar{D} + A\bar{C}\bar{B}$ \\
\end{equation}
\end{document}
